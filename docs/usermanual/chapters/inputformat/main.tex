\section{THOR Inputs}\label{sec:THOR_format}

The THOR transport solver has distinct user input formats for the following separate input files:
\begin{itemize}
    \item Standard Input File (Sec.~\ref{ch:inp:sec:stdinput}) - The primary input file to be run by THOR.
    All other input files will either be listed in this file, or assumed to be the default filenames as described in Section~\ref{ch:inp:sec:stdinput}.
    This is the only input file given to THOR by way of the command line.
    \item Mesh File (Sec.~\ref{ch:inp:sec:meshfile}) - File containing the physical 1st order tet mesh for the problem.
    \item Cross Section File (Sec.~\ref{ch:inp:sec:xsfile}) - File containing cross sections for the problem.
    \item Density Factor File (Sec.~\ref{ch:inp:sec:densfact}) - File containing the density factors for each adjustment of cross sections in each region.
    \item Initial Guess File (Sec.~\ref{ch:inp:sec:initguess}) - File containing the initial guess for the problem.
    \item Source File (Sec.~\ref{ch:inp:sec:srcfile}) - File containing the source for a fixed source problem.
    \item Inflow File (Sec.~\ref{ch:inp:sec:inflow}) - File containing the boundary condition inflow data for a fixed source problem.
\end{itemize}

This chapter describes the input formats of the THOR transport solver.

\section{THOR Standard Input Format}\label{ch:inp:sec:stdinput}

The following describes properties of the keyword based THOR input file:
\begin{itemize}
  \item Any keyword can appear in any order, but no keyword may appear multiple times.
  \item Every keyword has a default value, and THOR will echo a verbose form of the input at the beginning of the run, including all keywords and their values for the problem, whether they are set by the user or not.
  \item Whitespace is necessary between a parameter and the parameter values but is otherwise ignored.
  \item It is recommended that each parameter have its own line, however multiple parameters can be on the same line separated by semicolons (;).
  \item The user should ensure that line endings are UNIX text line endings, not Windows or Mac line endings.
  \item Whether multiple parameters are on the same line or not, the value immediately following the parameter is assumed to be that parameter's value.
  \item A line cannot contain more than 200 characters and most parameters must have all their values on the same line they reside, with exceptions outlined in the parameter descriptions, for some parameters that have a potentially large number of values (the only exception is \verb"region_map" at this time).
  \item Lines starting with an exclamation point, !, and blank lines will be ignored. Any data following an exclamation point on a used line will be ignored. This is equivalent to FORTRAN's comment style.
  \item The job name, \verb"<job_name>", is the input filename with extension removed if the extension is ``.in", ``.in", or ``.i"
\end{itemize}

\subsection{TYPE Card}
\begin{verbatim}
  type <prob_type>
\end{verbatim}
\begin{center}
  \begin{tabular}{| p{0.22\linewidth} | p{0.22\linewidth} | p{0.22\linewidth} | p{0.22\linewidth} |}
    \hline
    Keyword & Type & Options & Default \\ \hline
    \verb"type" & STRING & keig/fsrc & keig \\ \hline \hline
    \multicolumn{4}{| p{0.88\linewidth} |}{Description: Problem type. Either eigenvalue (keig) or fixed source (fsrc)}\\
    \hline
  \end{tabular}
\end{center}

\subsection{KEIGSOLVER Card}
\begin{verbatim}
  keigsolver <solver_type>
\end{verbatim}
\begin{center}
  \begin{tabular}{| p{0.22\linewidth} | p{0.22\linewidth} | p{0.22\linewidth} | p{0.22\linewidth} |}
    \hline
    Keyword & Type & Options & Default \\ \hline
    \verb"keigsolver" & STRING & pi/jfnk & pi \\ \hline \hline
    \multicolumn{4}{| p{0.88\linewidth} |}{Description: Solve type for keig. Either power iteration (pi) or Jacobian-Free Newton-Krylov (jfnk)}\\
    \hline
  \end{tabular}
\end{center}

\subsection{LAMBDA Card}
\begin{verbatim}
  lambda <spatial_order>
\end{verbatim}
\begin{center}
  \begin{tabular}{| p{0.22\linewidth} | p{0.22\linewidth} | p{0.22\linewidth} | p{0.22\linewidth} |}
    \hline
    Keyword & Type & Options & Default \\ \hline
    \verb"lambda" & INTEGER & - & 0 \\ \hline \hline
    \multicolumn{4}{| p{0.88\linewidth} |}{Description: Expansion order, negative number indicates reduced set}\\
    \hline
  \end{tabular}
\end{center}

\subsection{INFLOW Card}
\begin{verbatim}
  inflow <infl_spec>
\end{verbatim}
\begin{center}
  \begin{tabular}{| p{0.22\linewidth} | p{0.22\linewidth} | p{0.22\linewidth} | p{0.22\linewidth} |}
    \hline
    Keyword & Type & Options & Default \\ \hline
    \verb"inflow" & STRING & yes/no/\verb"<filename>" & no \\ \hline \hline
    \multicolumn{4}{| p{0.88\linewidth} |}{Description: If fixed inflow boundary conditions are provided for fsrc problems. If yes, then ``finflow.dat" is assumed to be the filename. If a string other than ``yes" or ``no" is given, then that string is assumed to be the filename.}\\
    \hline
  \end{tabular}
\end{center}

\subsection{PIACC Card}
\begin{verbatim}
  piacc <acc_method>
\end{verbatim}
\begin{center}
  \begin{tabular}{| p{0.22\linewidth} | p{0.22\linewidth} | p{0.22\linewidth} | p{0.22\linewidth} |}
    \hline
    Keyword & Type & Options & Default \\ \hline
    \verb"piacc" & STRING & errmode/none & none \\ \hline \hline
    \multicolumn{4}{| p{0.88\linewidth} |}{Description: Type of power iteration acceleration: none or error mode extrapolation}\\
    \hline
  \end{tabular}
\end{center}

\subsection{PAGE\_SWEEP Card}
\begin{verbatim}
  page_sweep <page_sweep_option>
\end{verbatim}
\begin{center}
  \begin{tabular}{| p{0.22\linewidth} | p{0.22\linewidth} | p{0.22\linewidth} | p{0.22\linewidth} |}
    \hline
    Keyword & Type & Options & Default \\ \hline
    \verb"page_sweep" & STRING & yes/no & no \\ \hline \hline
    \multicolumn{4}{| p{0.88\linewidth} |}{Description: If the sweep path is saved (no) or is paged to scratch file when not needed (yes)}\\
    \hline
  \end{tabular}
\end{center}

\subsection{PAGE\_REFL Card}
\begin{verbatim}
  page_refl <page_refl_option>
\end{verbatim}
\begin{center}
  \begin{tabular}{| p{0.22\linewidth} | p{0.22\linewidth} | p{0.22\linewidth} | p{0.22\linewidth} |}
    \hline
    Keyword & Type & Options & Default \\ \hline
    \verb"page_refl" & STRING & page/save/inner & save \\ \hline \hline
    \multicolumn{4}{| p{0.88\linewidth} |}{Description: If significant angular fluxes are paged to/from scratch file (page), stored (save), or discarded after completing inner iterations for a given group (inner)}\\
    \hline
  \end{tabular}
\end{center}

\subsection{PAGE\_IFLW Card}
\begin{verbatim}
  page_iflw <page_iflw_option>
\end{verbatim}
\begin{center}
  \begin{tabular}{| p{0.22\linewidth} | p{0.22\linewidth} | p{0.22\linewidth} | p{0.22\linewidth} |}
    \hline
    Keyword & Type & Options & Default \\ \hline
    \verb"page_iflw" & STRING & bygroup/all & all \\ \hline \hline
    \multicolumn{4}{| p{0.88\linewidth} |}{Description: If inflow information is loaded to memory completely (all) or for each group when required (bygroup)}\\
    \hline
  \end{tabular}
\end{center}

\subsection{KCONV Card}
\begin{verbatim}
  kconv <conv_criteria>
\end{verbatim}
\begin{center}
  \begin{tabular}{| p{0.22\linewidth} | p{0.22\linewidth} | p{0.22\linewidth} | p{0.22\linewidth} |}
    \hline
    Keyword & Type & Options & Default \\ \hline
    \verb"kconv" & REAL & - & $10^{-4}$ \\ \hline \hline
    \multicolumn{4}{| p{0.88\linewidth} |}{Description: Stopping criterion for eigenvalue}\\
    \hline
  \end{tabular}
\end{center}

\subsection{INNERCONV Card}
\begin{verbatim}
  innerconv <conv_criteria>
\end{verbatim}
\begin{center}
  \begin{tabular}{| p{0.22\linewidth} | p{0.22\linewidth} | p{0.22\linewidth} | p{0.22\linewidth} |}
    \hline
    Keyword & Type & Options & Default \\ \hline
    \verb"innerconv" & REAL & - & $10^{-4}$ \\ \hline \hline
    \multicolumn{4}{| p{0.88\linewidth} |}{Description: Stopping criterion for group flux during inner iteration}\\
    \hline
  \end{tabular}
\end{center}

\subsection{OUTERCONV Card}
\begin{verbatim}
  outerconv <conv_criteria>
\end{verbatim}
\begin{center}
  \begin{tabular}{| p{0.22\linewidth} | p{0.22\linewidth} | p{0.22\linewidth} | p{0.22\linewidth} |}
    \hline
    Keyword & Type & Options & Default \\ \hline
    \verb"outerconv" & REAL & - & $10^{-3}$ \\ \hline \hline
    \multicolumn{4}{| p{0.88\linewidth} |}{Description: Stopping criterion for group flux during outer/power iteration}\\
    \hline
  \end{tabular}
\end{center}

\subsection{MAXINNER Card}
\begin{verbatim}
  maxinner <num_iters>
\end{verbatim}
\begin{center}
  \begin{tabular}{| p{0.22\linewidth} | p{0.22\linewidth} | p{0.22\linewidth} | p{0.22\linewidth} |}
    \hline
    Keyword & Type & Options & Default \\ \hline
    \verb"maxinner" & INTEGER & - & 10 \\ \hline \hline
    \multicolumn{4}{| p{0.88\linewidth} |}{Description: Maximum number of inner iterations}\\
    \hline
  \end{tabular}
\end{center}

\subsection{MAXOUTER Card}
\begin{verbatim}
  maxouter <num_iters>
\end{verbatim}
\begin{center}
  \begin{tabular}{| p{0.22\linewidth} | p{0.22\linewidth} | p{0.22\linewidth} | p{0.22\linewidth} |}
    \hline
    Keyword & Type & Options & Default \\ \hline
    \verb"maxouter" & INTEGER & - & 100 \\ \hline \hline
    \multicolumn{4}{| p{0.88\linewidth} |}{Description: Maximum number of outer/power iterations}\\
    \hline
  \end{tabular}
\end{center}

\subsection{JFNK\_KRSZE Card}
\begin{verbatim}
  jfnk_krsze <krylov_space_size>
\end{verbatim}
\begin{center}
  \begin{tabular}{| p{0.22\linewidth} | p{0.22\linewidth} | p{0.22\linewidth} | p{0.22\linewidth} |}
    \hline
    Keyword & Type & Options & Default \\ \hline
    \verb"jfnk_krsze" & INTEGER & - & 25 \\ \hline \hline
    \multicolumn{4}{| p{0.88\linewidth} |}{Description: Maximum size of Krylov subspace during jfnk}\\
    \hline
  \end{tabular}
\end{center}

\subsection{JFNK\_MAXKR Card}
\begin{verbatim}
  jfnk_maxkr <num_iters>
\end{verbatim}
\begin{center}
  \begin{tabular}{| p{0.22\linewidth} | p{0.22\linewidth} | p{0.22\linewidth} | p{0.22\linewidth} |}
    \hline
    Keyword & Type & Options & Default \\ \hline
    \verb"jfnk_maxkr" & INTEGER & - & 250 \\ \hline \hline
    \multicolumn{4}{| p{0.88\linewidth} |}{Description: Maximum number of Krylov iterations}\\
    \hline
  \end{tabular}
\end{center}

\subsection{JFNK\_METHOD Card}
\begin{verbatim}
  jfnk_method <jfnk_method>
\end{verbatim}
\begin{center}
  \begin{tabular}{| p{0.22\linewidth} | p{0.22\linewidth} | p{0.22\linewidth} | p{0.22\linewidth} |}
    \hline
    Keyword & Type & Options & Default \\ \hline
    \verb"jfnk_method" & STRING & outer/flat/flat\_wds & flat \\ \hline \hline
    \multicolumn{4}{| p{0.88\linewidth} |}{Description: Type of jfnk formulation, see~\cite{refernce_manual} for details.}\\
    \hline
  \end{tabular}
\end{center}

\subsection{INITIAL\_GUESS Card}
\begin{verbatim}
  initial_guess <init_guess_spec>
\end{verbatim}
\begin{center}
  \begin{tabular}{| p{0.22\linewidth} | p{0.22\linewidth} | p{0.22\linewidth} | p{0.22\linewidth} |}
    \hline
    Keyword & Type & Options & Default \\ \hline
    \verb"initial_guess" & STRING & yes/no/\verb"<filename>" & no \\ \hline \hline
    \multicolumn{4}{| p{0.88\linewidth} |}{Description: If an initial guess file should be read. If yes, then ``initial\_guess.dat" is assumed to be the filename. If a string other than ``yes" or ``no" is given, then that string is assumed to be the filename.}\\
    \hline
  \end{tabular}
\end{center}

\subsection{RESTART\_OUT Card}
\begin{verbatim}
  restart_out <restart_out_spec>
\end{verbatim}
\begin{center}
  \begin{tabular}{| p{0.22\linewidth} | p{0.22\linewidth} | p{0.22\linewidth} | p{0.22\linewidth} |}
    \hline
    Keyword & Type & Options & Default \\ \hline
    \verb"restart_out" & STRING & yes/no/\verb"<filename>" & no \\ \hline \hline
    \multicolumn{4}{| p{0.88\linewidth} |}{Description: If a restart file should be written. If yes, then ``$<$job\_name$>$\_restart.out" is assumed to be the filename. If a string other than ``yes" or ``no" is given, then that string is assumed to be the filename.}\\
    \hline
  \end{tabular}
\end{center}

\subsection{IPITER Card}
\begin{verbatim}
  ipiter <num_iters>
\end{verbatim}
\begin{center}
  \begin{tabular}{| p{0.22\linewidth} | p{0.22\linewidth} | p{0.22\linewidth} | p{0.22\linewidth} |}
    \hline
    Keyword & Type & Options & Default \\ \hline
    \verb"ipiter" & INTEGER & - & 0 \\ \hline \hline
    \multicolumn{4}{| p{0.88\linewidth} |}{Description: Number of initial power iterations for jfnk}\\
    \hline
  \end{tabular}
\end{center}

\subsection{PRINT\_CONV Card}
\begin{verbatim}
  print_conv <print_conv_spec>
\end{verbatim}
\begin{center}
  \begin{tabular}{| p{0.22\linewidth} | p{0.22\linewidth} | p{0.22\linewidth} | p{0.22\linewidth} |}
    \hline
    Keyword & Type & Options & Default \\ \hline
    \verb"print_conv" & STRING & yes/no & no \\ \hline \hline
    \multicolumn{4}{| p{0.88\linewidth} |}{Description: If convergence monitor is written to file. If yes, then ``$<$job\_name$>$\_conv.convergence" is the convergence filename}\\
    \hline
  \end{tabular}
\end{center}

\subsection{DENSITY\_FACTOR Card}\label{ch:inp:sec:stdinput:ssec:densfact}
\begin{verbatim}
  density_factor <dens_fact_filename>
\end{verbatim}
\begin{center}
  \begin{tabular}{| p{0.22\linewidth} | p{0.12\linewidth} | p{0.22\linewidth} | p{0.32\linewidth} |}
    \hline
    Keyword & Type & Options & Default \\ \hline
    \verb"density_factor" & STRING & no/\verb"filename" & no \\ \hline \hline
    \multicolumn{4}{| p{0.88\linewidth} |}{Description: Density factor filename, or use no density factors (no).}\\
    \hline
  \end{tabular}
\end{center}

\subsection{EXECUTION Card}
\begin{verbatim}
  execution <exec_opt>
\end{verbatim}
\begin{center}
  \begin{tabular}{| p{0.22\linewidth} | p{0.22\linewidth} | p{0.22\linewidth} | p{0.22\linewidth} |}
    \hline
    Keyword & Type & Options & Default \\ \hline
    \verb"execution" & STRING & yes/no & yes \\ \hline \hline
    \multicolumn{4}{| p{0.88\linewidth} |}{Description: If yes problem is executed, if no then input is only read and checked.}\\
    \hline
  \end{tabular}
\end{center}

\subsection{MESH Card}
\begin{verbatim}
  mesh <mesh_filename>
\end{verbatim}
\begin{center}
  \begin{tabular}{| p{0.22\linewidth} | p{0.22\linewidth} | p{0.22\linewidth} | p{0.22\linewidth} |}
    \hline
    Keyword & Type & Options & Default \\ \hline
    \verb"mesh" & STRING & - & \verb"mesh.thrm" \\ \hline \hline
    \multicolumn{4}{| p{0.88\linewidth} |}{Description: Name of the mesh file.}\\
    \hline
  \end{tabular}
\end{center}

\subsection{SOURCE Card}
\begin{verbatim}
  source <source_filename>
\end{verbatim}
\begin{center}
  \begin{tabular}{| p{0.22\linewidth} | p{0.22\linewidth} | p{0.22\linewidth} | p{0.22\linewidth} |}
    \hline
    Keyword & Type & Options & Default \\ \hline
    \verb"source" & STRING & - & \verb"source.dat" \\ \hline \hline
    \multicolumn{4}{| p{0.88\linewidth} |}{Description: Name of the volumetric source file for fsrc problems.}\\
    \hline
  \end{tabular}
\end{center}

\subsection{FLUX\_OUT Card}
\begin{verbatim}
  flux_out <flux_filename>
\end{verbatim}
\begin{center}
  \begin{tabular}{| p{0.22\linewidth} | p{0.22\linewidth} | p{0.22\linewidth} | p{0.22\linewidth} |}
    \hline
    Keyword & Type & Options & Default \\ \hline
    \verb"flux_out" & STRING & - & \verb"<job_name>_flux.out" \\ \hline \hline
    \multicolumn{4}{| p{0.88\linewidth} |}{Description: Name of the THOR formatted output flux file}\\
    \hline
  \end{tabular}
\end{center}

\subsection{XS Card}
\begin{verbatim}
  xs <xs_filename>
\end{verbatim}
\begin{center}
  \begin{tabular}{| p{0.22\linewidth} | p{0.22\linewidth} | p{0.22\linewidth} | p{0.22\linewidth} |}
    \hline
    Keyword & Type & Options & Default \\ \hline
    \verb"xs" & STRING & - & \verb"xs.dat" \\ \hline \hline
    \multicolumn{4}{| p{0.88\linewidth} |}{Description: Name of the cross section file}\\
    \hline
  \end{tabular}
\end{center}

\subsection{VTK\_FLUX\_OUT Card}
\begin{verbatim}
  vtk_flux_out <vtk_flux_spec>
\end{verbatim}
\begin{center}
  \begin{tabular}{| p{0.22\linewidth} | p{0.22\linewidth} | p{0.22\linewidth} | p{0.22\linewidth} |}
    \hline
    Keyword & Type & Options & Default \\ \hline
    \verb"vtk_flux_out" & STRING & yes/no/\verb"<filename>" & no \\ \hline \hline
    \multicolumn{4}{| p{0.88\linewidth} |}{Description: If vtk flux file should be written. If yes, then ``$<$job\_name$>$\_flux.vtk" is assumed to be the filename. If a string other than ``yes" or ``no" is given, then that string is assumed to be the filename.}\\
    \hline
  \end{tabular}
\end{center}

\subsection{VTK\_MAT\_OUT Card}
\begin{verbatim}
  vtk_mat_out <vtk_mat_spec>
\end{verbatim}
\begin{center}
  \begin{tabular}{| p{0.22\linewidth} | p{0.22\linewidth} | p{0.22\linewidth} | p{0.22\linewidth} |}
    \hline
    Keyword & Type & Options & Default \\ \hline
    \verb"vtk_mat_out" & STRING & yes/no/\verb"<filename>" & no \\ \hline \hline
    \multicolumn{4}{| p{0.88\linewidth} |}{Description: If vtk material file should be written. If yes, then ``$<$job\_name$>$\_mat.vtk" is assumed to be the filename. If a string other than ``yes" or ``no" is given, then that string is assumed to be the filename.}\\
    \hline
  \end{tabular}
\end{center}

\subsection{VTK\_REG\_OUT Card}
\begin{verbatim}
  vtk_reg_out <vtk_reg_spec>
\end{verbatim}
\begin{center}
  \begin{tabular}{| p{0.22\linewidth} | p{0.22\linewidth} | p{0.22\linewidth} | p{0.22\linewidth} |}
    \hline
    Keyword & Type & Options & Default \\ \hline
    \verb"vtk_reg_out" & STRING & yes/no/\verb"<filename>" & no \\ \hline \hline
    \multicolumn{4}{| p{0.88\linewidth} |}{Description: If vtk region file should be written. If yes, then ``$<$job\_name$>$\_reg.vtk" is assumed to be the filename. If a string other than ``yes" or ``no" is given, then that string is assumed to be the filename.}\\
    \hline
  \end{tabular}
\end{center}

\subsection{VTK\_SRC\_OUT Card}
\begin{verbatim}
  vtk_src_out <vtk_src_spec>
\end{verbatim}
\begin{center}
  \begin{tabular}{| p{0.22\linewidth} | p{0.22\linewidth} | p{0.22\linewidth} | p{0.22\linewidth} |}
    \hline
    Keyword & Type & Options & Default \\ \hline
    \verb"vtk_src_out" & STRING & yes/no/\verb"<filename>" & no \\ \hline \hline
    \multicolumn{4}{| p{0.88\linewidth} |}{Description: If vtk source file should be written. If yes, then ``$<$job\_name$>$\_src.vtk" is assumed to be the filename. If a string other than ``yes" or ``no" is given, then that string is assumed to be the filename.}\\
    \hline
  \end{tabular}
\end{center}

\subsection{CARTESIAN\_MAP\_OUT Card}
\begin{verbatim}
  cartesian_map_out <cartesia_map_filename>
\end{verbatim}
\begin{center}
  \begin{tabular}{| p{0.22\linewidth} | p{0.22\linewidth} | p{0.1\linewidth} | p{0.34\linewidth} |}
    \hline
    Keyword & Type & Options & Default \\ \hline
    \verb"cartesian_map_out" & STRING & - & \verb"<job_name>\_cartesian\_map.out" \\ \hline \hline
    \multicolumn{4}{| p{0.88\linewidth} |}{Description: Name of the THOR formatted Cartesian map output file}\\
    \hline
  \end{tabular}
\end{center}

\subsection{PRINT\_XS Card}
\begin{verbatim}
  print_xs <print_xs_opt>
\end{verbatim}
\begin{center}
  \begin{tabular}{| p{0.22\linewidth} | p{0.22\linewidth} | p{0.22\linewidth} | p{0.22\linewidth} |}
    \hline
    Keyword & Type & Options & Default \\ \hline
    \verb"print_xs" & STRING & yes/no & no \\ \hline \hline
    \multicolumn{4}{| p{0.88\linewidth} |}{Description: If cross sections are echoed to standard output.}\\
    \hline
  \end{tabular}
\end{center}

\subsection{PNORDER Card}
\begin{verbatim}
  pnorder <pn_order>
\end{verbatim}
\begin{center}
  \begin{tabular}{| p{0.22\linewidth} | p{0.22\linewidth} | p{0.22\linewidth} | p{0.22\linewidth} |}
    \hline
    Keyword & Type & Options & Default \\ \hline
    \verb"pnorder" & INTEGER & - & 0 \\ \hline \hline
    \multicolumn{4}{| p{0.88\linewidth} |}{Description: Spherical harmonics order used for scattering in code.}\\
    \hline
  \end{tabular}
\end{center}

\subsection{QDTYPE Card}
\begin{verbatim}
  qdtype <quad_tp>
\end{verbatim}
\begin{center}
  \begin{tabular}{| p{0.22\linewidth} | p{0.1\linewidth} | p{0.34\linewidth} | p{0.22\linewidth} |}
    \hline
    Keyword & Type & Options & Default \\ \hline
    \verb"qdtype" & STRING & levelsym/legcheb/\verb"<filename>" & levelsym \\ \hline \hline
    \multicolumn{4}{| p{0.88\linewidth} |}{Description: Quadrature type: level-symmetric, Legendre-Chebyshev, or read from file if a filename is given (read from file not currently supported).}\\
    \hline
  \end{tabular}
\end{center}

\subsection{QDORDER Card}
\begin{verbatim}
  qdorder <quad_ord>
\end{verbatim}
\begin{center}
  \begin{tabular}{| p{0.22\linewidth} | p{0.22\linewidth} | p{0.22\linewidth} | p{0.22\linewidth} |}
    \hline
    Keyword & Type & Options & Default \\ \hline
    \verb"qdorder" & INTEGER & - & 4 \\ \hline \hline
    \multicolumn{4}{| p{0.88\linewidth} |}{Description: Order of the angular quadrature.}\\
    \hline
  \end{tabular}
\end{center}

\subsection{CARTESIAN\_MAP Card}
\begin{verbatim}
  cartesian_map <cart_map_spec>
\end{verbatim}
\begin{center}
  \begin{tabular}{| p{0.14\linewidth} | p{0.26\linewidth} | p{0.38\linewidth} | p{0.1\linewidth} |}
    \hline
    Keyword & Type & Options & Default \\ \hline
    \verb"cartesian_map" & STRING/REAL (9 entries) & no/xmin, xmax, nx, ymin, ymax, ny, zmin, zmax, nz & no \\ \hline \hline
    \multicolumn{4}{| p{0.88\linewidth} |}{Description: Sets up an overlayed Cartesian mesh that fluxes and reactions rates are averaged over. The Cartesian mesh is defined by the minimum and maximum coordinates for each direction (x, y, z) and number of subdivisions between.}\\
    \hline
  \end{tabular}
\end{center}

\subsection{POINT\_VALUE\_LOCATIONS Card}
\begin{verbatim}
  point_value_locations <points>
\end{verbatim}
\begin{center}
  \begin{tabular}{| p{0.22\linewidth} | p{0.22\linewidth} | p{0.22\linewidth} | p{0.22\linewidth} |}
    \hline
    Keyword & Type & Options & Default \\ \hline
    \verb"point_value_locations" & STRING/REAL (3 N) & - & no \\ \hline \hline
    \multicolumn{4}{| p{0.88\linewidth} |}{Description: Allows extraction of flux values at user provided points. N is the number of points, (x,y,z) coordinates of N points, x1 y1 z1 x2 y2...}\\
    \hline
  \end{tabular}
\end{center}

\subsection{REGION\_MAP Card}
\begin{verbatim}
  region_map <region_maps>
\end{verbatim}
\begin{center}
  \begin{tabular}{| p{0.12\linewidth} | p{0.22\linewidth} | p{0.42\linewidth} | p{0.12\linewidth} |}
    \hline
    Keyword & Type & Options & Default \\ \hline
    \verb"region_map" & STRING/INTEGER & no/reg1 mat1 reg2 mat2 reg3 mat3...  & no \\ \hline \hline
    \multicolumn{4}{| p{0.88\linewidth} |}{Description: Mapping from region id to cross section id. Region ids are an integer assigned to to each tetrahedral element that are used to group elements into regions or blocks (see Sect.~\ref{sec:mesh_format}). Cross section ids are indices that identify sets of cross sections provided in the cross section input file (see Sect.~\ref{sec:cross_section_format}). If no map is provided, then the mapping is assumed to be one to one, i.e. region 4 maps to cross section material 4, region 8 maps to cross section material 8, etc. The ``region\_map" card can have entries on multiple lines.}\\
    \hline
  \end{tabular}
\end{center}

The \verb"region_map" card is best illustrated for an example.
Let us assume that we have regions -1,4,7,19 and we want to assign the cross section materials as follows:
\begin{verbatim}
    -1 -> 12
     4 -> 1
     7 -> 1
     19 -> 3
\end{verbatim}
Then the \verb"region_map" card is given by:
\begin{verbatim}
region_map -1 12 4 1 7 1 19 3
\end{verbatim}
or, since the \verb"region_map" can be specified on multiple lines:
\begin{verbatim}
region_map
 -1 12
  4 1
  7 1
  19 3
\end{verbatim}

\subsection{Legacy Data Cards}

The following cards specify data for deprecated features.
Unless legacy features are being used, this data is not necessary and will be ignored.
\begin{center}
  \begin{tabular}{| p{0.250\linewidth} | p{0.102\linewidth} | p{0.176\linewidth} | p{0.102\linewidth} | p{0.250\linewidth} |}
    \hline
    Keyword & Type & Options & Default & Legacy Application \\
    \hline
    \hline
    \verb"ngroups" & INTEGER & - & 1 &  Old XS format\\ \hline
    \multicolumn{5}{| p{0.88\linewidth} |}{Description: Number of energy groups in cross section file.}\\
    \hline
    \hline
    \verb"pnread" & INTEGER & - & 0 &  Old XS format \\ \hline
    \multicolumn{5}{| p{0.88\linewidth} |}{Description: Spherical harmonics expansion provided in cross section file.}\\
    \hline
    \hline
    \verb"upscattering" & STRING & yes/no & yes &  Old XS format \\ \hline
    \multicolumn{5}{| p{0.88\linewidth} |}{Description: Read upscattering data from cross section file or ignore it.}\\
    \hline
    \hline
    \verb"multiplying" & STRING & yes/no & yes &  Old XS format \\ \hline
    \multicolumn{5}{| p{0.88\linewidth} |}{Description: If the cross section file contains fission information.}\\
    \hline
    \hline
    \verb"scatt_mult_included" & STRING & yes/no & yes &  Old XS format \\ \hline
    \multicolumn{5}{| p{0.88\linewidth} |}{Description: If the cross section file scattering data includes the $2 l + 1$ multiplier or not.}\\
    \hline
  \end{tabular}
\end{center}

\section{THOR Mesh Format}\label{ch:inp:sec:meshfile}

Line 1: number of vertices
\vspace{2mm}

\noindent Line 2: number of elements
\vspace{2mm}

\noindent Line 3: unused enter 1
\vspace{2mm}

\noindent Line 4: unused enter 1
\vspace{2mm}

\noindent Block 1: vertex coordinates, number of lines = number of vertices; each line is as follows: \\
\verb"vertex_id x-coordinate y-coordinate z-coordinate"
\vspace{2mm}

\noindent Block 2: region and source id assignments, number of lines = number of elements; each line is as follows: \\
\verb"element_id region_id source_id"\\
For setting up Monte Carlo on the tet mesh, this block can be ignored.
\vspace{2mm}

\noindent Block 3: element descriptions, the vertex\_ids that form each element. Number of lines = number of elements; each line is as follows: \\
\verb"element_id vertex_id1 vertex_id2 vertex_id3 vertex_id4"
\vspace{2mm}

\noindent Next line: number of boundary face edits
\vspace{2mm}

\noindent Block 4: boundary face descriptions. All exterior faces associated with their boundary condition id, number if lines = number of boundary face edits; each line is as follows: \\
\verb"element_id local_tetrahedron_face_id boundary_condition_id"\\
Explanation: local\_tetrahedron\_face\_id: natural local id of tetrahedron’s face which is the id of the vertex opposite to this face. Note: indexed 0-3.
boundary\_condition\_id:
value = 0: vacuum BC
value = 1: reflective BC
value = 2: fixed inflow
\vspace{2mm}

\noindent Next line: number of adjacency list entries
\vspace{2mm}

\noindent Block 5: adjacency list, number of lines = number of adjacency list entries; each line is as follows: \\
\verb"element_id face_id neighbor_id neighbor_face_id"\\
Explanation: The element\_id is the current element. The neighbor across the face indexed by face\_id has the element id neighbor\_id and the its own local index for the said common face is neighbor\_face\_id.

\section{THOR Cross Section Format}\label{ch:inp:sec:xsfile}

Lines starting with an exclamation point, !, and blank lines will be ignored.
Any data following an exclamation point on a used line will be ignored. This is equivalent to FORTRAN's comment style.
An example of the format is given in \verb"<thor_dir>/THOR/examples/c5g7.xs".
The following is the order of the data as it appears in the cross section file:
\begin{verbatim}
Line 1: THOR_XS_V1 <num_mats> <G> <L>

Line 2: energy_group_boundary_1... energy_group_boundary_G

Block 1: Each entry in this block contains cross sections for a single material.
Each block contains  (L+1)*G+5 lines. There are num_mats blocks.

Entry line 1: id <material_id>  name <material_name>
Entry line 2: fission_spectrum_1 fission_spectrum_2... fission_spectrum_G
Entry line 3: Sigma_f_1 Sigma_f_2 Sigma_f_3... Sigma_f_G
Entry line 4: nu_bar_1 nu_bar_2... nu_bar_G
Entry line 5: Sigma_t_1 Sigma_t_2... Sigma_t_G
Entry line 6: sig_scat_{0, 1->1} sig_scat_{0, 2->1}... sig_scat_{0, G->1}
Entry line 7: sig_scat_{0, 1->2} sig_scat_{0, 2->2}... sig_scat_{0, G->2}
        :
Entry line G+5: sig_scat_{0, 1->G} sig_scat_{0, 2->G}... sig_scat_{0, G->G}
Entry line G+6: sig_scat_{1, 1->1} sig_scat_{1, 2->1}... sig_scat_{1, G->1}
Entry line G+7: sig_scat_{1, 1->2} sig_scat_{1, 2->2}... sig_scat_{1, G->2}
        :
Entry line 2*G+5: sig_scat_{1, 1->G} sig_scat_{1, 2->G}... sig_scat_{1, G->G}
        :
Entry line L*G+6: sig_scat_{L, 1->G} sig_scat_{L, 2->G}... sig_scat_{L, G->G}
        :
\end{verbatim}

\begin{itemize}
\item \verb"num_mats" = Total number of cross section materials.
\item \verb"G" = Total number of energy groups.
\item \verb"L" = Scattering expansion order.
\item \verb"energy_group_boundary_g": Currently unused, can be filled with 0s. Upper bound of energy group g. The assumption is that the energy structure is the same for all materials.
\item \verb"material_id" = Index of the material. Used in identifying the material and region mapping.
\item \verb"material_name" = Name of the material. Not used except in output for the user to keep track of materials.
\item \verb"fission_spectrum_g": Fraction of neutrons born in fission that appear in energy group g ($\chi$).
\item \verb"Sigma_f_g": Fission cross section in group g ($\Sigma_f$ NOT $\nu\Sigma_f$).
\item \verb"nu_bar_g": average number of neutrons released by fission caused by a neutron in energy group g ($\nu$).
\item \verb"Sigma_t_g": total cross section in energy group g ($\Sigma_t$).
\item \verb"sig_scat_{l, g'->g}": l-th Legendre polynomial moment of the scattering cross section from group g’ to g ($\Sigma_{s,l,g'\rightarrow g}$). The (2 * l + 1) factor may be included in the value of the cross section or not, THOR can handle both cases. It needs to be specified separately every time.
\end{itemize}

\section{THOR Density Factor Format}\label{ch:inp:sec:densfact}

THOR density factors are used to adjust cross sections in the transport calculation.
The first line in the file contains the \verb"<adj_type>", specifying whether the data contained within is \verb"volumes" or \verb"dens_facts" for the data.
If \verb"volumes" is specified, then the \verb"adjustment" values are actual volumes of the regions pre-meshing.
The meshed volume is then divided by the exact volume and the resulting ratio is the scaling factor for the cross sections in that region in THOR.
If \verb"dens_facts" is specified, then the \verb"adjustment" values are assumed to be the actual scaling factors for the cross sections for the specified region in THOR.
The following describes the density factor format for THOR:
\begin{verbatim}
Line 1: <adj_type>
Line 2: <region_number> <adjustment>
Line 3: <region_number> <adjustment>
              :
\end{verbatim}

\section{THOR Initial Guess Format}\label{ch:inp:sec:initguess}

THOR can read in an initial guess file for the transport calculation.
This file is expected to be in unformatted FORTRAN binary.
Typically the user need not worry about the structure of this file, the form is identical to that of the restart output file in Section~\ref{ch:out:sec:restart}, and in fact the expectation is that the user will only use an initial guess file from that THOR generated restart data.

\section{THOR Source Format}\label{ch:inp:sec:srcfile}

Fixed source problems in THOR should include a file specifying an external source.
This file has a fixed format but lines may be commented with an exclamation point at the start of the line, or the ends of lines may be commented starting with an exclamation point.
The order of the data for each source ID is all groups for a given source spatial/angular moment are on each line, then all spatial moments for a given angular moment are given line after line, which then repeats for each angular moment.
This description is repeated below:
\begin{verbatim}
Line 1: THOR_SRC_V1 n_src_id n_ang_mom n_spat_mom

Block 1: The data in this block contains the source description for a single source ID.
Cells are assigned sources by ID in the mesh file.
Each block contains  n_ang_mom*n_spat_mom+1 lines. There are n_src_id blocks.

Entry line 1: src_id
Entry line 2: Q_{1,1,1} Q_{1,1,2}... Q_{1,1,G}
Entry line 3: Q_{1,2,1} Q_{1,2,2}... Q_{1,2,G}
                  :
Entry line n_spat_mom+1: Q_{n_spat_mom,1,1} Q_{n_spat_mom,1,2}... Q_{n_spat_mom,1,G}
Entry line n_spat_mom+2: Q_{1,2,1} Q_{1,2,2}... Q_{1,2,G}
                  :
Entry line n_ang_mom*n_spat_mom+1: Q_{n_spat_mom,n_ang_mom,1}...
                Q_{n_spat_mom,n_ang_mom,2} Q_{n_spat_mom,n_ang_mom,G}
\end{verbatim}
The sources given are the actual sources as they will be used in the calculation, not multipliers of the base 0th moment (indexed as 1) source.
If higher order moment data is given than the problem that THOR is solving (i.e. if a 0th spatial order and 2nd angular problem is being solved and \verb"n_spat_mom>1"/\verb"n_ang_mom>3" source is given), then THOR will ignore the higher order source data and it wont be used.
This feature allows for arbitrarily high order source specification that can be used in alternate versions of the same problem to determine the necessity of a certain spatial/angular expansion order without requiring the user remake the source for each calculation.