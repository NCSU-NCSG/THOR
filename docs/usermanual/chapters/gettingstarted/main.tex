\section{Obtaining THOR}

This section describes how to obtain THOR.
Please contact the code owner~\href{mailto:yyazmy@ncsu.edu}{Yousry Azmy} to be added to the project's membership.

Once a user has been added to the projects membership, they can navigate to their desired installation directory and clone THOR from the NCSU GitHub repository using the following command:
\begin{verbatim}
  >> git clone https://github.ncsu.edu/NCSU-Rad-Transport/THOR.git
\end{verbatim}
Alternatively, a user who will communicate frequently with THOR's github repository can link their computer's ssh keys with their GitHub account and clone THOR directly by issuing the follwoing command:
\begin{verbatim}
  >> git clone git@github.ncsu.edu:NCSU-Rad-Transport/THOR.git
\end{verbatim}

\section{Obtaining lapack dependencies}
THOR depends on certain lapack routines. These are provided with THOR as a submodule. The lapack submodule can be initialized by:
\begin{verbatim}
    >> git submodule update --init
\end{verbatim}
The lapack submodule is not expected to change at all. However, if it does, the THOR repository keeps track of the associated version of the lapack repository, so the user may run:
\begin{verbatim}
    >> git submodule update
\end{verbatim}
to obtain the latest lapack submodule. If as expected lapack hasn't changed an empty line will be displayed.

\section{Compiling THOR}
This section describes how to compile THOR and its dependencies.
The first step is to compile the lapack dependency.
To this end, navigate to the installation scripts using (where \verb"<thor_dir>" is the directory THOR was cloned into):
\begin{verbatim}
    >> cd <thor_dir>/contrib/scripts
\end{verbatim}
Edit the file \verb"make.inc" to specify the MPI Fortran compiler available on the local machine (if {\tt gfortran} and {\tt mpich} are being used, there is no need to make any changes).
Also, if necessary, enter command line that modify the environment to enable the compilation process to find the path to required executables; these typically have the form \verb">> load module pathname", where \verb"pathname" is a directory on the local computer where these necessary executables reside.
Execute the \verb"build_lapack.sh" script by (first command may not be necessary, it only ensures that \verb"build_lapack.sh" is executable):
\begin{verbatim}
    >> chmod +x build_lapack.sh
    >> ./build_lapack.sh <n>
\end{verbatim}
where \verb"<n>" is the number of processors.
%At this point it must be provided even if it is \verb"1".
For example, on Idaho National Laboratory's Sawtooth HPCthe compiler is set in \verb"make.inc" via the statement \verb"FORTRAN = mpif90", and the environment is  modified with the command line
\begin{verbatim}
    >> module load mvapich2/2.3.3-gcc-8.4.0".
\end{verbatim}
A successful lapack build will conclude the scrolled output on the screen with a table of the form:
\begin{verbatim}
                        -->   LAPACK TESTING SUMMARY  <--
                Processing LAPACK Testing output found in the TESTING directory
SUMMARY                 nb test run     numerical error         other error
================        ===========     =================       ================
REAL                    1291905         0       (0.000%)        0       (0.000%)
DOUBLE PRECISION        1292717         0       (0.000%)        0       (0.000%)
COMPLEX                 749868          0       (0.000%)        0       (0.000%)
COMPLEX16               749588          1       (0.000%)        1       (0.000%)

--> ALL PRECISIONS      4084078         1       (0.000%)        1       (0.000%)
\end{verbatim}
The lapack build may conclude with:
\begin{verbatim}
make[2]: Leaving directory '<thor_dir>/contrib/lapack/TESTING/EIG'
NEP: Testing Nonsymmetric Eigenvalue Problem routines
./EIG/xeigtstz < nep.in > znep.out 2>&1
make[1]: *** [Makefile:464: znep.out] Error 139
make[1]: Leaving directory '<thor_dir>/contrib/lapack/TESTING'
make: *** [Makefile:43: lapack_testing] Error 2
\end{verbatim}
These errors indicate that the system did not have enough memory allocated to lapack to complete the entirety of the testing suite.
This is typically not a concern and if these are the sole errors the user is free to continue on to the next step.
The correctness of THOR and the lapack linkage can later be verified with the regression tests if the user so desires.

Now, THOR can be compiled. Navigate to the cloned THOR folder, and then to the source folder within it:
\begin{verbatim}
    >> cd \verb"<thor_dir>"/THOR/src
\end{verbatim}
and, as before, edit the file \verb"Makefile" to utilize the available MPI Fortran compiler and if necessary modify the environment to enable \verb"make" to locate the compiler (again, for {\tt gfortran} and {\tt mpich}, no changes are necessary).
Then type:
\begin{verbatim}
    >> make
\end{verbatim}
Successful compilation of THOR will conclude with the line:
\begin{verbatim}
    mv ./thor-1.0.exe ../
\end{verbatim}
The THOR executable (named in the above line) can be found here:
\begin{verbatim}
    >> ls <thor_dir>/THOR/
\end{verbatim}
that should produce:
\begin{verbatim}
    doc  examples  hello_world  scripts  src   thor-1.0.exe  unit
\end{verbatim}

\section{Running THOR for the first time}
Navigate to the \verb"hello_world" directory:
\begin{verbatim}
    >> cd <thor_dir>/THOR/hello_world
\end{verbatim}
Check the content of this folder:
\begin{verbatim}
    >> ls
\end{verbatim}
It should show the following files:
\begin{verbatim}
    >> ls
    hello_world.in hello_world.o hello_world.thrm hello_world.xs
\end{verbatim}
These files have the following significance:
\begin{itemize}
    \item \verb"hello_world.in" is a sample input file to THOR. This file is used to execute THOR.
    \item \verb"hello_world.thrm" is the corresponding mesh file that is referenced within \verb"hello_world.in". At this point, it is only important that it is present and has the proper THOR mesh format. Creation of THOR mesh files is covered later in this manual.
    \item \verb"hello_world.xs" is the corresponding cross section file, also referenced within \verb"hello_world.in", and again at this point, it is only important that it is present.
    \item \verb"hello_world.o" is the corresponding output file created by redirecting THOR's standard output. This file can be used to compare THOR's printed output with what it should be upon correct termination of this run.
\end{itemize}
THOR is invoked at a minimum with the executable name and the standard input file that is specified after the \verb"-i" modifier.
\begin{verbatim}
    >> ../thor-1.0.exe -i hello_world.in
\end{verbatim}
For parallel execution type:
\begin{verbatim}
    >> mpiexec -n <n> ../thor-1.0.exe -i hello_world.in
\end{verbatim}
where \verb"<n>" is the number of processors.
Several files should have been created:
\begin{itemize}
    \item \verb"hello_world.flux"
    \item \verb"hello_world.fluxeven"
    \item \verb"hello_world.fluxodd"
    \item \verb"hello_world.in_out.csv"
    \item \verb"intermediate_output_even.dat" \item \verb"intermediate_output_odd.dat"
\end{itemize}
The significance of these files will be discussed later.
THOR's standard output should start with a banner and conclude with:
\begin{verbatim}
 --------------------------------------------------------
    Region averaged reaction rates
 --------------------------------------------------------

 -- Region --   0 Volume=   1.500000E+01

    Group          Flux       Fission    Absorption      Fiss Src
        1  9.515584E-01  1.284604E+00  8.564026E-01  1.284604E+00
 Total     9.515584E-01  1.284604E+00  8.564026E-01  1.284604E+00

 --------------------------------------------------------
    Execution of THOR completed successfully
 --------------------------------------------------------
\end{verbatim}

\section{Pre/post Processors}

\subsection{Setting up THOR\_MESH\_Generator}
THOR mesh generator converts \textit{exodus} and \textit{gmsh}
mesh formats to THOR's native mesh format. It also permits
uniform refinement of meshes provided in exodus files. Conversion from \textit{exodus} format and uniform refinement uses the
\textit{libmesh}~\cite{libMeshPaper} \textit{meshtool}. Therefore,
\textit{libmesh} has to be set up first. To this end, navigate to the \verb"scripts" directory:
\begin{verbatim}
    >> cd /home/<usr>/projects/THOR/contrib/scripts
\end{verbatim}
and execute \verb"build_libmesh.sh":
\begin{verbatim}
    >> chmod +x build_libmesh.sh
    >> ./build_libmesh.sh <n>
\end{verbatim}
where the first command makes \verb"build_libmesh.sh" executable (if it is not already) and \verb"<n>" is the number of processors. It must be provided even if it is simply 1. Executing this script may take a long time to complete installing \textit{libmesh}, however, it will show progress on the screen. If the git-clone command in the \verb"build_libmesh.sh" script does not work, replace it with the command:
\begin{verbatim}
git clone https://github.com/libMesh/libmesh.git
\end{verbatim}
Finally, the \verb"THOR_LIBMESH_DIRECTORY" environment variable has to be set. This environment variable must point to the directory that the \verb"meshtool-opt" executable is located. For the standard installation, one should execute:
\begin{verbatim}
    >> export THOR_LIBMESH_DIRECTORY=/home/<usr>/projects/THOR/contrib/libmesh/build
\end{verbatim}

The next step is to make the THOR\_MESH\_GENERATOR application. Navigate to its source folder:
\begin{verbatim}
    >> cd /home/<usr>/projects/THOR/pre-processors/THOR_Mesh_Generator/src
\end{verbatim}
and type:
\begin{verbatim}
    >> make
\end{verbatim}
The exectutable
\begin{verbatim}
    /home/<usr>/projects/THOR/pre-processors/THOR_Mesh_Generator/Thor_Mesh_Generator.exe
\end{verbatim}
should have been created.

\begin{verbatim}
***************************************************************
The tests as described below did not execute as described.
Instead, I did the following and still execution of the tests
did not work properly:

1. In ~/PROJECTS/THOR/pre-processors/THOR_Mesh_Generator:
ln -s Thor_Mesh_Generator\_MP.exe Thor_Mesh_Generator.exe

2. In ~/PROJECTS/THOR/pre-processors/THOR\_Mesh_Generator/scripts:
chmode u+x test\_all.sh

3. ./test\_all.sh
This ran but did not give the output below and reported execution
errors. It is not clear if these reported errors are part of the
testing since some of the cases are labeled Bad, or the error
indicates erroneous installation of libmesh.
***************************************************************
\end{verbatim}


To ensure that the THOR\_MESH\_GENERATOR application compiled correctly, execute the regression tests. Change directory to:
\begin{verbatim}
    >> cd /home/<usr>/projects/THOR/pre-processors/THOR_Mesh_Generator/scripts
\end{verbatim}
and execute:
\begin{verbatim}
    >> python run_thor_tests.py
\end{verbatim}
You should see screen output similar to this:
\begin{verbatim}
--------------------------------------------------------------------------------
Test  1 tests/bad_gmesh_non_tet_element:bad_gmesh_non_tet_element success
Test  2 tests/bad_gmesh_no_elements_block:bad_gmesh_no_elements_block success
Test  3 tests/homogeneous_domain:homogeneous success
Test  4 tests/homogeneous_domain:homogeneous_from_exodus success
Test  5 tests/homogeneous_domain:homogeneous_r1 success
Test  6 tests/homogeneous_domain:homogeneous_from_exodus_r1 success
Test  7 tests/bad_gmesh_no_nodes_block:bad_gmesh_no_nodes_block success
Test  8 tests/bad_gmesh_no_format_block:bad_gmesh_no_format_block success
Test  9 tests/bad_gmesh_non_tri_face:bad_gmesh_non_tri_face success
Test  10 tests/convert_old_to_new_THOR:convert_old_to_new_THOR success
Test  11 tests/unv_sphere_in_shell_in_box:unv_sphere_in_shell_in_box success
Test  12 tests/Basic_Cube_Mesh_test:basic_cube_mesh success
Test  13 tests/split_hex_and_prism:split_hex success
Test  14 tests/split_hex_and_prism:split_prism success
Test  15 tests/bad_unv_sphere_in_shell_in_box_no_2411:bad_unv_sphere_in_shell_in_box_no_2411 success
--------------------------------------------------------------------------------
Successes:  15            Failures:  0
\end{verbatim}
All or at least the vast majority of tests should pass, so \verb"Failures" should be close to zero.