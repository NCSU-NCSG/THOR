The THOR transport solver has distinct output formats for the following separate output files:
\begin{itemize}
    \item Log file - All standard output to the terminal in THOR is echoed to a log file.
      The filename for the log file is the filename of the input file with the extension \verb".log" appended.
      If the input files has the \verb".inp", \verb".in", or \verb".i" extension, then that extension will be removed before the \verb".log" extension is added.
      i.e. for the input file \verb"thisfile.in", the log file will be titled \verb"thisfile.log".
    \item Restart file (Sec.~\ref{ch:out:sec:restart}) - A binary file containing an initial guess to restart the calculation if it is interrupted or if stronger convergence is later desired.
    \item Flux file (Sec.~\ref{ch:out:sec:flux}) - A file containing the final flux distribution.
    \item VTK flux file (Sec.~\ref{ch:out:sec:vtk}) - A vtk file containing the final flux distribution.
    \item VTK material file (Sec.~\ref{ch:out:sec:vtk}) - A vtk file containing the spatial material information.
    \item VTK region file (Sec.~\ref{ch:out:sec:vtk}) - A vtk file containing the region spatial information.
    \item VTK source file (Sec.~\ref{ch:out:sec:vtk}) - A vtk file containing the external source information for fixed source problems.
    \item Cartesian map file (Sec.~\ref{ch:out:sec:cart}) - A file containing the flux and reaction rate results from the overlaid Cartesian map specified by the user.
\end{itemize}

\section{THOR Restart Output}\label{ch:out:sec:restart}

THOR can read output a restart file for the transport calculation.
This file is an unformatted FORTRAN binary file.
Typically the user need not worry about the structure of this file, its sole intended use is as a restart file (as the initial guess) for a future calculation for either increased convergence or in the event that the original calculation was interrupted before completion.

\section{THOR Flux Output}\label{ch:out:sec:flux}

THOR can output a flux file containing the group-wise scalar flux data from the final solution to the transport calculation performed.
The first line is \verb"numels", the number of tet elements in the problem.
Each line following the first has a cell's volume followed by the group-wise flux starting from group 1 to group G.
The format is as follows:
\begin{verbatim}
Line 1: numels
Line 2: el_vol_1 flux_1_1 flux_1_2... flux_1_G
Line 3: el_vol_2 flux_2_1 flux_2_2... flux_2_G
                    :
Line numels+1: el_vol_numels flux_numels_1 flux_numels_2... flux_numels_G
\end{verbatim}

\section{THOR VTK Outputs}\label{ch:out:sec:vtk}

THOR can output four types of VTK files for use with \href{https://www.paraview.org/}{ParaView} or other software used to analyze or visualize VTK formatted data.
An example of visualization of the flux VTK file is shown in Chapter~\ref{ch:tutorials} for the two THOR tutorial problems.
The VTK files THOR can output includes a flux file with the final flux data from the solution to the transport problem, a material file with material mapping from the THOR mesh/input files, a region file with region mapping from the THOR mesh file, and a source file with source mapping from the THOR mesh/input files.

\section{THOR Cartesian Output}\label{ch:out:sec:cart}

THOR can output data from an overlaid Cartesian that the user can specify, as described in Section~\ref{ch:inp:sec:inp:ssec:cartmap}.
The data given is the spatially averaged flux over the Cartesian grid mesh.
The file also contains the spatially averaged total, absorption, scattering, and fission reaction rates in addition to the spatially averaged fission source production rate.
The format is as follows:
\begin{verbatim}
Line 1: column labels
Line 2: cell1_x_idx cell1_y_idx cell1_z_idx cell1_flux cell1_total_rr cell1_abs_rr
          cell1_scatt_rr cell1_fiss_rr cell1_fiss_prod
Line 2: cell2_x_idx cell2_y_idx cell2_z_idx cell2_flux cell2_total_rr cell2_abs_rr
          cell2_scatt_rr cell2_fiss_rr cell2_fiss_prod
                            :
\end{verbatim}