THOR currently includes two tutorials to guide new users through the process of creating a mesh with Gmsh, using the Mesh Converter to make it a THOR mesh, and running THOR using that mesh.
The first tutorial is the Godiva tutorial, creating a model of the bare Godiva critical experiment sphere, meshing it, and running it with THOR.
This tutorial demonstrates basic THOR problem creation and running concepts.

The second tutorial is the BeRP tutorial, creating a model of the BeRP ball with 3 inches of polyethylene reflector surrounding the BeRP ball.
This tutorial demonstrates a more advanced problem involving multiple materials and regions.

Since both systems are physically equivalent to one-dimensional spherical problems, the symmetry is taken advantage of and the problems are modeled using a one/eighth model with reflective boundary conditions to be equivalent to the full spheres for the tutorial.
In the tutorial folders, \verb"<thor_dir>/THOR/examples/Godiva_tutorial" and \verb"<thor_dir>/THOR/examples/BeRP_tutorial", versions of these tutorials with full spheres, half spheres, and quarter spheres are also included along with reference results.

\section{Godiva Tutorial}

Godiva is an un-shielded, pulsed, nuclear burst reactor. It is essentially a homogeneous sphere of highly enriched uranium with a diameter of $30$ cm, that was operated by inserting a piston of fissile material~\cite{Godiva1961}.
In this tutorial the critical benchmark configuration described in Ref.~\cite{GodivaBenchmark} is considered. The geometry that is modeled by THOR is a homogeneous sphere of radius $8.7407$ cm discretized by tetrahedra similar to Fig.~\ref{fig:godiva_coarse}.
The energy domain is discretized with six energy groups, and cross sections are provided by~\cite{GodivaBenchmark}.

\begin{figure}[th]
  \includegraphics[width=1.0\textwidth]{chapters/tutorials/figures/godiva_coarse.png}
  \caption{Coarse mesh for Godiva problem}
  \label{fig:godiva_coarse}
\end{figure}

This tutorial first explains how a tetrahedral mesh is created for the Godiva problem, then the cross sections data input is discussed, and finally the standard input to THOR is covered.
The input files discussed below for the Godiva tutorial are located in:
\begin{verbatim}
    >> <thor_dir>/THOR/examples/Godiva_tutorial
\end{verbatim}

\subsection{Godiva Mesh}\label{ch:tuts:sec:godiva:ssec:mesh}

The workflow described here is suitable if the user has access to a compatible version of \href{https://gmsh.info/}{Gmsh}.
Any version 4 Gmsh should work, but the example specifically performed here was done using Gmsh version 4.10.1.

Begin by navigating to the location of the Godiva Gmsh geometry file, \verb"godiva_octant.geo", which is found in \verb"<thor_dir>/THOR/examples/Godiva_tutorial/mesh_create/".
Opening the file in a text editor, it can be observed that the model is created by removing the negative portions of each direction from a sphere centered at the origin.
For more details on creating original Gmsh inputs, see the Gmsh \href{https://gmsh.info/doc/texinfo/gmsh.html}{reference manual}.

Open \verb"godiva_octant.geo" in Gmsh and run the ``3D'' command from the ``Mesh'' dropdown menu under ``Modules''.
The mesh should be generated and now become visible in the GUI.
Now, select the ``Save'' command from the same ``Mesh'' dropdown to save the generated mesh to \verb"godiva_octant.msh".
This mesh may be compared to the provided \verb"godiva_octant_msh.ref", however they may differ if the versions differ or if optimization of the mesh is employed.

The gmsh file \verb"godiva_octant.msh" is converted to THOR's native mesh format by executing the Mesh Converter with the command line:
\begin{verbatim}
  >> <thor_dir>/THOR/pre-processors/Mesh_Converter/Mesh_Converter.exe convert_godiva.in
     -bc 1 0 1 0 1 0
\end{verbatim}
Note that since we are modeling the fully positive octant of the sphere we are setting all of the flat negative faces to be reflective (see Section~\ref{ch:getstart:sec:preproc:subsec:meshconv} for more details).
After successful completion of the conversion, the following printout should appear:
\begin{verbatim}
---------------------Reading in gmsh:

---------------------Calculating Adjacencies:

---------------------Outputting thrm file:

---------------------Calculating volumes:

Region 5 volume:   3.3099604624050352E+02
Region 5 equivalent Radius:   4.2911933629822876E+00
Total System Volume:   3.3099604624050352E+02
Equivalent Radius:   4.2911933629822876E+00

**********************************************************************************
**********************************************************************************
**********************************************************************************
**************************THOR mesh converter sucessful.**************************
--------------- Output written to godiva_octant.msh_out.thrm
\end{verbatim}

The file \verb"godiva_octant.msh_out.thrm" should result from this execution for use by THOR.
This mesh may be compared to the provided \verb"godiva_octant_thrm.ref", which it should match if \verb"godiva_octant.msh" matches \verb"godiva_octant_msh.ref".
Notice that the given volume for Region 5 (the Godiva eighth of a sphere as seen in \verb"godiva_octant.geo") is 330.996~cm$^3$, but the actual octant volume for the Godiva sphere is 349.653~cm$^3$.
The ratio of the actual volume to the meshed volume is then 1.056366, which will come in handy later.
This concludes the mesh generation step for this tutorial.

\subsection{Cross section data}

The THOR cross section file for the Godiva benchmark is provided by:
\begin{verbatim}
    <thor_dir>/THOR/examples/Godiva_tutorial/run_files/godiva.xs
\end{verbatim}
THOR uses a custom cross section format that is explained in detail in Sec.~\ref{ch:inp:sec:xsfile}.

At the end of Section~\ref{ch:tuts:sec:godiva:ssec:mesh}, it was observed that there was a discrepancy in the volume of the Godiva mesh compared to the original problem.
To preserve material mass, the cross sections must be altered by increasing them by a factor of 1.056366.
In THOR, the user need not alter the cross sections themselves to make this adjustment.
Instead, THOR will automatically adjust reaction and material attenuation calculations by a given density factor.
By default, this factor is 1.0, which will lead to use of the original cross sections unaltered.
However, the user may specify density factors in a density factor file, described in Section~\ref{ch:inp:sec:densfact}.
For this tutorial, this density factor adjustment is provided by:
\begin{verbatim}
    <thor_dir>/THOR/examples/Godiva_tutorial/run_files/godiva_octant.dens
\end{verbatim}


\subsection{THOR input file and executing THOR}

The THOR input file is:
\begin{verbatim}
   <thor_dir>/THOR/examples/Godiva_tutorial/run_files/godiva_octant.inp
\end{verbatim}
THOR uses a keyword-based input that is listed in Sect.~\ref{sec:THOR_format}.
The Godiva tutorial input file is verbose and some parameters are ignored as they are not relevant to the problem.
Upon running THOR, a verbose form of the input will always be echoed, and ignored parameters will be highlighted.
\begin{verbatim}
type                  keig
keigsolver            pi
lambda                0
inflow                no
piacc                 errmode
page_sweep            no
page_refl             save
page_iflw             all
kconv                 1e-8
innerconv             1e-12
outerconv             1e-7
maxinner              4
maxouter              5000
jfnk_krsze            25
jfnk_maxkr            250
jfnk_method           flat
initial_guess         no
restart_out           no
ipiter                0
print_conv            yes
density_factor        fromfile godiva_octant.dens
execution             yes
mesh                  ./godiva_octant_thrm.ref
source                source.dat
flux_out              no
xs                    ./godiva.xs
vtk_flux_out          no
vtk_mat_out           no
vtk_reg_out           no
vtk_src_out           no
cartesian_map_out     no
print_xs              no
ngroups               1
pnorder               0
pnread                0
upscattering          yes
multiplying           yes
scatt_mult_included   yes
qdtype                levelsym
qdorder               4
cartesian_map         no
point_value_locations no
region_map            5 1
\end{verbatim}

To solve the problem, the user should move \verb"godiva_octant.msh_out.thrm" to the input file location
\begin{verbatim}
  <thor_dir>/THOR/examples/Godiva_tutorial/run_files/
\end{verbatim}
After navigating to the Godiva tutorial input file location, the Godiva tutorial is solved with THOR via the command line:
\begin{verbatim}
  >> <thor_dir>/THOR/thor-1.0.exe godiva_octant.inp
\end{verbatim}

Completion of execution of the Godiva tutorial is indicated by the printout:
\begin{verbatim}
--------------------------------------------------------
   Execution of THOR completed successfully
--------------------------------------------------------
\end{verbatim}

THOR provides the following output that is discussed in this tutorial:
\begin{itemize}
    \item The final estimate of the multiplication factor is printed under ``Execution Summary'', ``Final eigenvalue''. In this case the value is $0.9611$. This is not close to critical because the mesh that
    is created does not conserve the volume of the critical sphere. When creating a tetrahedral mesh,
    Cubit places the nodes on the sphere's surface, but that necessarily leads to the volume of the discretized geometry to be smaller than the original volume. In this case, the computational volume
    is $2742.4\text{ cm}^3$, while the actual volume is $2767.8\text{ cm}^3$.
    \item A summary of group-wise, region-averaged reaction rates is provided for each region identifier separately under ``Region averaged reaction rates''. The volume of each region, and group-wise fluxes, fission, absorption, and fission source rates are listed.
    \item Two vtk formatted files, \verb"flux.vtk" contains spatial flux maps, and \verb"mat.vtk" contains the material map. These files can be opened with the paraview post-processing tool that is available \href{https://www.paraview.org/download/}{here}.
\end{itemize}

The reaction rate summary is given by:
\begin{verbatim}
--------------------------------------------------------
    Region averaged reaction rates
--------------------------------------------------------

-- Region --   1 Volume=   2.742401E+03

   Group          Flux       Fission    Absorption      Fiss Src
       1  8.424948E-01  1.399140E-01  4.946487E-02  1.399140E-01
       2  1.573081E+00  2.333728E-01  9.509448E-02  2.333728E-01
       3  9.814645E-01  1.379655E-01  5.968679E-02  1.379655E-01
       4  1.631588E+00  2.215631E-01  1.006868E-01  2.215631E-01
       5  1.198341E+00  1.915944E-01  9.089592E-02  1.915944E-01
       6  1.792724E-01  4.652155E-02  2.413688E-02  4.652155E-02
Total     6.406241E+00  9.709313E-01  4.199657E-01  9.709313E-01
\end{verbatim}

The results can be improved greatly by enforcing volume conservation on the Cubit mesh. This is accomplished manually in this tutorial by changing the radius in the Cubit journal file to .